\documentclass[a4paper,11pt, fontset=fandol]{ctexart}
\usepackage{graphicx}
\usepackage{geometry}
\usepackage{amssymb}
\usepackage{float}
\usepackage{amssymb}
\usepackage{changepage}
\usepackage{listings}
\usepackage{amsmath}
\usepackage{fontspec}
\usepackage{xunicode}
\usepackage{xltxtra}
\usepackage{titlesec}
\usepackage{mathptmx}
\usepackage{enumerate}
\usepackage{booktabs}
\usepackage{xeCJK}
\usepackage{caption}

\setmainfont{Times Roman}

\newcommand*{\TNR}{\CJKfamily{TNR}}
\newcommand{\yihao}{\fontsize{28pt}{\baselineskip}\selectfont}
\newcommand{\erhao}{\fontsize{21pt}{\baselineskip}\selectfont}
\newcommand{\sanhao}{\fontsize{15.75pt}{\baselineskip}\selectfont}
\newcommand{\sihao}{\fontsize{14pt}{\baselineskip}\selectfont}
\newcommand{\xiaosihao}{\fontsize{12pt}{\baselineskip}\selectfont}
\newcommand{\wuhao}{\fontsize{10.5pt}{\baselineskip}\selectfont}
\newcommand{\xiaowuhao}{\fontsize{9pt}{\baselineskip}\selectfont}
\titleformat*{\section}{\sihao\bfseries\songti}
\titleformat*{\subsection}{\xiaosihao\bfseries\songti}


\geometry{right=2.54cm,left=2.54cm,top=3.18cm,bottom=3.18cm}
\setlength{\parskip}{0.5em}
\begin{document}
\begin{center}
\textbf{\sanhao 所学即所为?地方官员教育背景与产业发展} \\
\kaishu \wuhao 李宏扬 \\
\songti (浙江大学公共管理学院城市发展与管理系,杭州,310058)\\
\end{center}
\heiti \noindent 摘要: \songti \\
\heiti \noindent 关键词:\songti 


\begin{center}
\textbf{\TNR\sihao What you learn is what you do? Local officials' educational background and industrial development} \\
\wuhao LI Hongyang \\
(Department of Urban Development and Management, School of Public Affairs, Zhejiang University, Hangzhou, 310058) 
\end{center}
\noindent \textbf{Abstract}: \\
\noindent \textbf{Keywords}: 

\newpage
\tableofcontents
\newpage
\songti

\section{引言}

\section{文献综述}

\section{理论框架与研究假设}

\section{数据说明与模型设计}
\subsection{数据说明}
本研究综合使用地方官员任职履历信息与地级市政府工作报告文本,构建教育背景与产业政策关注度之间的分析框架。地方官员数据来自结构化的市级人事履历数据库,数据表包含144个字段,核心变量包括官员的任职年份、地区、职务、学历、毕业院校与专业等。研究关注官员的教育背景信息,尤其关注记录官员所学专业的原始文本数据。政府工作报告数据来源于公开发布的地级市年度政府工作报告文本,该类报告由地方市长在人大会议上公开发布,系统总结上一年度地方政府的主要工作和下一阶段的政策规划,文本内容结构严谨,涵盖产业发展、城市规划、民生保障、生态环境等多个板块。地级市的年度政府工作报告文本平均长度在1.5万字以上,内容包含地区主要产业布局与调整、年度重点项目与财政投资方向、技术创新与人才引育政策和城乡发展规划与生态目标等。

\section{实证结果与分析}

\section{结论与政策建议}





\end{document}